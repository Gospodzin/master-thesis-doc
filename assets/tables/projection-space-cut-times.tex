\begin{table}
	\centering
	\begin{tabular}{ | c | c | c | c |  c | c | c | c | c | }
		\hline
		$ name $ & $ n_{div} $ & $ n_p $ & $ \varepsilon $ & $ \mu $ & $ T $ & $ T_{h1} $ & $ T_{h2} $ & $ T_{h1}+T_{h2} $\\ \hline
		 sequoia  & 0 & 1 & 5000 & 4 & 0.91 & 0.35 & 0.34 & 0.69 \\ \hline
		 birch  & 0 & 1 & 4000 & 4 & 2.10 & 0.58 & 0.59 & 1.17 \\ \hline
		 cup98  & 36 & 37 & 10 & 4 & 13.01 & 8.92 & 2.91 & 11.83 \\ \hline
		covtype & 9 & 5 & 40 & 4 & 611.63 & 194.83 & 126.66 & 321.49 \\ \hline
	\end{tabular}
	\caption{Czasy grupowania z użyciem algorytmu DBSCAN z metodą projekcji na wymiarze $ n_p $. Zbiory danych $ D $ podzielono na podzbiory $ h_1 $, dla punktów takich, że współrzędna $ n_{div} $ jest mniejsza od mediany, oraz $ h_2 $ dla takich, że jest większa. Wymiary zostały dobrane na podstawie bezwzględnego średniego odchylenia, tak, że $ n_p $ jest wymiarem o największej wartości, a $ n_{div} $ jest wymiarem o drugiej największej wartości. $ T $ - czas grupowania całego zbioru, $ T_{h1} $ - czasy grupowania $ h_1 $, $ T_{h2} $ - czas grupowania $ h_2 $}\label{qscan:projection-space-cut-times}
\end{table}