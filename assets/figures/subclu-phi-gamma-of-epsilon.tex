\begin{figure}
	\centering
	\begin{tikzpicture}
		\newcommand{\varDataFile}{assets/data/subclu-phi-gamma-of-epsilon.dat}
		\begin{axis}[
				xmin=0, ymin=0,
				width=\varImgWidth,
				xlabel= $ \varepsilon $,
				extra y ticks={0.16129},
				extra y tick labels={$\frac{5}{2^5-1}$}]
			\addplot[blue, mark=*] table[x={eps}, y={phi}]{\varDataFile};
			\addplot[red, mark=diamond*] table[x={eps},y={gamma}]{\varDataFile};
			\legend{$ \phi $, $ \gamma $}
		\end{axis}
	\end{tikzpicture}
	\caption{Wartości $ \phi $ oraz $ \gamma $ w zależności od $ \varepsilon $. Dla każdej z wartości $ \varepsilon $ algorytm SUBCLU został wykonany z parametrem $ \mu=10 $. Zbiór danych i algorytm grupowania podprzestrzeni jak w opisie \myhyperref{fig:odc:subclu-gamma-of-n}{rysunku} dla $ n = 5 $. }
	\label{fig:odc:subclu-phi-gamma-of-epsilon}
\end{figure}