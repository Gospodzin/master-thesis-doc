\begin{figure}
	\begin{minipage}[b]{.52\linewidth}
		\centering
		\begin{tikzpicture}
			\begin{axis}[
					width=0.961538461\linewidth,
					xlabel=liczba punktów,
					ylabel=czas wykonania,
					y unit=\si{\s},
				]
				\addplot table {assets/data/odc.dat};
			\end{axis}
		\end{tikzpicture}
		\subcaption{wydajne jedno-wymiarowe grupowanie}
	\end{minipage}%
	\begin{minipage}[b]{.48\linewidth}
	\centering
		\begin{tikzpicture}
			\begin{axis}[
					width=1.04166666666\linewidth,
					xlabel=liczba punktów,
					ylabel=czas wykonania,
					y unit=\si{\s},
				]
				\addplot table {assets/data/dbscan-1d.dat};
			\end{axis}
  	\end{tikzpicture}
		\subcaption{TIDBSCAN}
	\end{minipage}
	\caption{Czas grupowania jednowymiarowych danych za pomocą wydajnego jedno-wymiarowego grupowania oraz TIDBSCAN w zależności od wielkości zbioru danych. Algorytmy wykonane jak w opisie \myhyperref{odc:odc-vs-dbscan-ratio}{Rysunku}.}\label{odc:odc-and-dbscan}
\end{figure}