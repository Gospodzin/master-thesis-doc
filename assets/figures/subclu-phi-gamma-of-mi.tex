\begin{figure}
	\centering
	\begin{tikzpicture}
		\newcommand{\varDataFile}{assets/data/subclu-phi-gamma-of-mi.dat}
		\begin{axis}[
				xmin=0,
				width=\varImgWidth,
				xlabel= $ \mu $]
			\addplot[blue, mark=*] table[x={mi}, y={phi}]{\varDataFile};
			\addplot[red, mark=diamond*] table[x={mi},y={gamma}]{\varDataFile};
			\legend{$ \phi $, $ \gamma $}
		\end{axis}
	\end{tikzpicture}
	\caption{Wartości $ \phi $ oraz $ \gamma $ w zależności od $ \mu $. Dla każdej z wartości $ \varepsilon $ algorytm SUBCLU został wykonany z parametrem $ \varepsilon=200 $. Zbiór danych i algorytm grupowania podprzestrzeni jak w opisie \myhyperref{fig:odc:subclu-gamma-of-n}{rysunku} dla $ n = 5 $.}
	\label{fig:odc:subclu-phi-gamma-of-mi}
\end{figure}