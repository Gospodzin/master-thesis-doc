\begin{figure}
	\centering
	\begin{tikzpicture}
		\newcommand{\varDataFile}{assets/data/subclu-gamma-of-n.dat}
		\begin{axis}[xmin=1, ymin=0, samples=50, width=\varImgWidth, xlabel= $n$-wymiarowość zbioru danych]
			\addplot[blue, mark=*, domain=1:10, samples=10]{x/(2^x-1)};
			\addplot[red, mark=square*] table[x={dimensions}, y={eps-200-mi-10}]{\varDataFile};
			\addplot[black, mark=triangle*] table[x={dimensions}, y={eps-200-mi-50}]{\varDataFile};
			\legend{$ \frac{n}{2^n-1} $,
				{$ \gamma $ dla $ \varepsilon=200 $, $ \mu=10 $},
				{$ \gamma $ dla $ \varepsilon=200 $, $ \mu=50 $}}
		\end{axis}
	\end{tikzpicture}
	\caption{Wartość $ \gamma $ oraz dolne ograniczenie w zależności od wymiarowości zbioru danych. Dla każdej z $ n $ wymiarowych przestrzeni algorytm SUBCLU został wykonany ze wskazanymi parametrami $ \mu $ oraz $ \varepsilon $ na 5000 losowo wygenerowanych punktach z rozkładu równomiernego na zbiorze \mbox{$ D \subseteq X^n \,|\, X = \set{ 0...9999 } $}. Do grupowania podprzestrzeni wykorzystano TI-DBSCAN \cite{tidbscan} z jednym punktem referencyjnym \mbox{$ max(D) = r \in X^n \,|\, \forall v \in D$, $\forall i \in \set{1...n}$, $\exists w \in D : r_i \ge v_i \land r_i = w_i$.}}
	\label{fig:odc:subclu-gamma-of-n}
\end{figure}