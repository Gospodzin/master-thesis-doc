Istniejące algorytmy grupowania można podzielić ze względu na to, jak definiują pojęcie grupy. Jedną z klas algorytmów grupowania są algorytmy grupowania gęstościowego, które znajdują grupy na podstawie lokalnej gęstości punktów, co zgadza się z intuicyjnym pojęciem grupy.

Część algorytmów grupowania gęstościowego opiera się na dzieleniu przestrzeni na siatkę komórek, gdzie grupy określane są na podstawie uśrednionej gęstości wewnątrz tychże komórek. O ile takie podejście jest proste i wydajne, to nie jest dokładne, a wyniki grupowania mogą zależeć od rozmieszczenia siatki. Algorytm DBSCAN reprezentuje podejście, które rozwiązuje te problemy. Co więcej, wyniki są deterministyczne z dokładnością do tak zwanych punktów brzegowych grup.\par
Definicja grupy zaproponowana w DBSCAN okazała się na tyle skuteczna, że powstały liczne algorytmy pochodne opierające się na tej samej lub podobnej definicji grupy. Takim algorytmem jest SUBCLU. SUBCLU należy do klasy algorytmów grupowania w podprzestrzeniach, co oznacza, że jest \mbox{w stanie} wyznaczać grupy w podzbiorach zbioru współrzędnych grupowanych punktów.\par
W tym rozdziale zostaną przedstawione dwa wspomniane wcześniej algorytmy: DBSCAN i SUBCLU.