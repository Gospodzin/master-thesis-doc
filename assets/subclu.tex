\section{SUBCLU}

SUBCLU \cite{subclu} jest algorytmem grupowania gęstościowego bazującym na \linebreak DBSCAN. Pozwala na wykrycie grup znajdujących się we wszystkich podprzestrzeniach grupowanego zbioru danych, gdzie podprzestrzeń jest rozumiana jako podzbiór zbioru współrzędnych grupowanych punktów. Grupowanie \mbox{w podprzestrzeniach} jest szczególnie przydatne, gdy mamy do czynienia \mbox{z silnie} wysokowymiarowymi danymi. Wysokowymiarowe punkty są \mbox{z reguły} rzadko rozmieszczone w przestrzeni, a niektóre atrybuty mogą zaszumiać grupy istniejące w podprzestrzeniach, dlatego grupowanie algorytmem DBSCAN wysokowymiarowych danych może nie dawać sensownych wyników.

\subsection{Definicje}
Definicje zawarte w podrozdziale o DBSCAN zostają zapożyczone i zmodyfikowane, tak aby umożliwić operacje w podprzestrzeni $ S $. Projekcja punktu $ p $ na podprzestrzeń $ S $ będzie oznaczana $ \s{pi}_S(p) $. Zakładam, że poddany grupowaniu jest $ n $ wymiarowy zbiór punktów $ D $ i wartości $ \varepsilon $, $ \mu $ są ustalone.
\smallskip
\smallskip

\definition{\s{d:spaceandsubspace}}\newline
 Przestrzenią jest zbiór wszystkich wymiarów $ A=\set{a_1,\dots,a_n} $ zbioru $ D $. Podprzestrzenią jest każdy podzbiór przestrzeni wymiarów $ S \subseteq A $.
\smallskip

\definition{\s{d:eps-otoczenie s} (ang. $ \varepsilon $-neighbourhood in subspace)}\newline
Zbiór punktów $ V \subseteq \s{D} $ jest $ \varepsilon $-otoczeniem punktu $ p\in D $ w \mbox{podprzestrzeni $ S $} ($ \s{N*}^S_{D,\varepsilon}(p) $), jeśli dla każdego punktu $ q \in V $ projekcja $ \pi_S(q) $ znajduje się w odległości nie większej niż $ \varepsilon $ od projekcji $ \pi_S(p) $.
\begin{equation}
	\s{N*}^S_{D,\varepsilon}(p) = \set{q \in D | d(\pi_s(p), \pi_s(q)) \le \varepsilon} \land p\in D
\end{equation}
Odnosząc się do $ \varepsilon $-otoczenia w podprzestrzeni $ S $, będę stosował uproszczony zapis: \textit{\s{eps-otoczenie s}$^S$}.
\smallskip

\definition{\s{d:punkt rdzeniowy w podprzestrzeni} (ang. core point in subspace)}\newline
Punkt $ p \in D $ jest punktem rdzeniowym w podprzestrzeni $ S $ ($ \s{core*}^S_{D,\varepsilon,\mu}(p) $), jeśli liczba punktów w $ \varepsilon $-otoczeniu$^S $ punktu $ p $ jest większa lub równa $ \mu $.
\begin{equation}
	\s{core*}^S_{D,\varepsilon,\mu}(p) \iff |N^S_{D,\varepsilon}(p)| \ge \mu \land p\in D
\end{equation}
Odnosząc się do punktu rdzeniowego w podprzestrzeni $ S $, będę stosował uproszczony zapis: \textit{\s{punkt rdzeniowy s}$^S$}.
\smallskip

\definition{\s{d:dirreach s} (ang. direct density reachability in subspace)} \newline
Punkt $ p\in D $ jest bezpośrednio gęstościowo osiągalny z punktu $ q\in D $ \mbox{w podprzestrzeni $ S $} ($ \s{dirreach*}^S_{D,\varepsilon,\mu}(p, q) $), jeśli $ q $ jest punktem rdzeniowym \mbox{i $ p $ należy} do $ \varepsilon $-otoczenia$^S$ punktu $ q $.
\begin{equation}
	\s{dirreach*}^S_{D,\varepsilon,\mu}(p, q) \iff p \in N^S_{D,\varepsilon}(q) \land core^S_{D,\varepsilon,\mu}(q)
\end{equation}
Odnosząc się do bezpośredniej gęstościowej osiągalności w podprzestrzeni $ S $, będę stosował uproszczony zapis: \textit{\s{dirreach s}$^S$}.
\smallskip

\definition{\s{d:punkt brzegowy w podprzestrzeni} (ang. edge point in subspace)} \newline
Punkt $ p\in D $ jest punktem brzegowym w podprzestrzeni $ S $ ($ edge^S_{D,\varepsilon,\mu}(p) $), jeśli nie jest punktem rdzeniowym$^S$ i jest bezpośrednio gęstościowo osiągalny$^S$ z innego punktu $ q $.
\begin{equation}
	\s{edge*}^S_{D,\varepsilon,\mu}(p) \iff \neg core^S_{D,\varepsilon,\mu}(p) \land dirreach^S_{D,\varepsilon,\mu}(p,q)
\end{equation}
Odnosząc się do punktu brzegowego w podprzestrzeni $ S $, będę stosował uproszczony zapis: \textit{\s{punkt brzegowy s}$^S$}.
\smallskip

\definition{\s{d:reach s} (ang. density reachability in subspace)} \newline
Punkt $ p_1\in D $ jest gęstościowo osiągalny z punktu $ p_n\in D $ w podprzestrzeni $ S $ \linebreak($ reach^S_{D,\varepsilon,\mu}(p_1, p_n) $), jeśli istnieje ciąg punktów $ p_1,\dots,p_n \in D $, taki że każdy punkt tego ciągu, poza ostatnim jest, bezpośrednio gęstościowo osiągalny$^S$ z punktu następnego.
\begin{equation}
	\exists_{p_1,\dots,p_n}\,\forall_{i\in\set{1,\dots,n-1}}\,dirreach^S_{D,\varepsilon,\mu}(p_i, p_{i+1}) \implies \s{reach*}^S_{D,\varepsilon,\mu}(p_1, p_n)	 
\end{equation}
Odnosząc się do gęstościowej osiągalności w podprzestrzeni $ S $, będę stosował uproszczony zapis: \textit{\s{reach s}$^S$}.
\smallskip

\definition{\s{d:grupa w podprzestrzeni} (ang. cluster in subspace)} \newline
Zbiór punktów $ C $ jest grupą w podprzestrzeni $ S $ w zbiorze punktów $ D $, jeśli zawiera się w $ D $ i jest zbiorem wszystkich punktów gęstościowo osiągalnych$^S$ z dowolnego punktu rdzeniowego$^S$. 
\begin{equation}
C = \set{p \in D | \s{reach}^S_{D,\varepsilon,\mu}(p, q)} \land core^S_{D,\varepsilon,\mu}(q)
\end{equation}
Odnosząc się do grupy w podprzestrzeni $ S $, będę stosował uproszczony zapis: \textit{\s{grupa s}$^S$}.
\smallskip

\definition{\s{d:punkt szumu w podprzestrzeni} (ang. noise point in subspace)} \newline
Punkt $ p\in D $ jest punktem szumu w podprzestrzeni $ S $ ($ \s{noise}_{D,\varepsilon,\mu}(p) $), jeśli nie jest punktem rdzeniowym$^S$ oraz nie jest bezpośrednio gęstościowo osiągalny$^S$ z żadnego punktu $q \in D$. 
\begin{equation}
noise^S_{D,\varepsilon,\mu}(p) \iff \neg core^S_{D,\varepsilon,\mu}(p) \land \forall_{q\in D}\,\neg dirreach^S_{D,\varepsilon,\mu}(p, q)
\end{equation}
Odnosząc się do punktu szumu w podprzestrzeni $ S $, będę stosował uproszczony zapis: \textit{\s{punkt szumu s}$^S$}.
\smallskip

\definition{\s{d:conset} (ang. density connected set in subspace)}\newline
Zbiór punktów $ V \subseteq D $ jest gęstościowo połączonym zbiorem w podprzestrzeni $ S $ ($ \s{conset}^S_{D,\varepsilon, \mu}(V) $), jeśli istnieje taki punkt $ p \in D $, że każdy punkt $ q \in V $ jest gęstościowo osiągalny$^S$ z punktu $ p $.
\begin{equation}
	conset^S_{D,\varepsilon, \mu}(V) \iff 
	\exists_{p\in D}
	\forall_{q\in V}
	\,reach^S_{D,\varepsilon,\mu}(q, p)
\end{equation}
Odnosząc się do gęstościowo połączonego zbioru w podprzestrzeni $ S $, będę stosował uproszczony zapis: \textit{gęstościowo połączony zbiór$^S$}. 

Każda grupa$ ^S $ jest jednocześnie gęstościowo połączonym zbiorem$^S $, ale nie każdy gęstościowo połączony zbiór$^S $ $ V $ jest grupą$^S $, ponieważ $ V $ nie gwarantuje, że zawiera wszystkie punkty gęstościowo osiągalne$ ^S $ z punktu $ p $ takiego, że każdy punkt $ q\in V $ jest gęstościowo osiągalny$ ^S $ z $ p $.
\smallskip

\definition{\s{d:consetmon} (ang. monotonicity of density connectivity)}\newline
Monotonicznością gęstościowej łączności nazywa się własność, która oznacza, że jeśli istnieje gęstościowo połączony zbiór$^S$ $ V z$, to $ V $ jest też gęstościowo połączonym zbiorem$^T$ takim, że $ T \subseteq S $.
\begin{equation}
	T\subseteq S \land conset^S_{D,\varepsilon,\mu}(V) \implies conset^T_{D,\varepsilon,\mu}(V)
\end{equation}

\subsection{Algorytm grupowania w podprzestrzeniach}
Koncepcyjnie najprostszym rozwiązaniem problemu znalezienia grup we wszystkich podprzestrzeniach zbioru D jest zastosowanie algorytmu DBSCAN do znalezienia grup w każdej z podprzestrzeni oddzielnie. W ten sposób można znaleźć wszystkie grupy istniejące w podprzestrzeniach zbioru $ D $. Istotnym problemem jest jednak liczba istniejących podprzestrzeni \mbox{zbioru $ D $}. Jeśli przyjmiemy, że zbiór $ D $ jest $ n $ wymiarowy, to zbiór $ D $ posiada $ 2^n $ podprzestrzeni. Grupowanie algorytmem DBSCAN trzeba przeprowadzić w każdej z podprzestrzeni poza $ 0 $ wymiarową podprzestrzenią, więc znalezienie grup we wszystkich podprzestrzeniach wymaga $ 2^n-1 $ grupowań algorytmem DBSCAN. Jest to istotne ograniczenie uniemożliwiające zastosowanie tej metody dla wysokowymiarowych danych.

Okazuje się, że można istotnie ograniczyć liczbę punktów i podprzestrzeni, w których jest przeprowadzane grupowanie. Autorzy SUBCLU \cite{subclu} wykorzystali w tym celu własność monotoniczności gęstościowej łączności. Monotoniczność gęstościowej łączności wynika z faktu, że dla pary punktów $ p $ i $ q $ odległość między projekcjami $ \pi_S(p) $ i $ \pi_S(q) $ jest zawsze większa bądź równa od odległości między projekcjami $ \pi_T(p) $ i $ \pi_T(q) $, gdzie $ T \subseteq S $. 
\begin{equation}\label{eq:d-vs-d-projected}
\forall_{T\subseteq S}\,d(\pi_S(p),\pi_S(q)) \geq d(\pi_T(p), \pi_T(q))
\end{equation}
Z \myhyperref{eq:d-vs-d-projected}{wyrażenia} wynikają zależności, które uzasadniają monotoniczność gęstościowej łączności.
\begin{equation}
	\begin{array}{l}
		\big(d(\pi_S(p),\pi_S(q)) \le \varepsilon \implies d(\pi_T(p), \pi_T(q)) \le \varepsilon\big) \implies\\
		\big(N^S_{D,\varepsilon}(p) \subseteq N^T_{D,\varepsilon}(p) \land \big(core^S_{D,\varepsilon,\mu}(p) \implies core^T_{D,\varepsilon,\mu}(p)\big)\big) \implies \\
		\big(dirreach^S_{D,\varepsilon,\mu}(p, q) \implies dirreach^T_{D,\varepsilon,\mu}(p, q)\big) \implies \\
		\big(reach^S_{D,\varepsilon,\mu}(p, q) \implies reach^T_{D,\varepsilon,\mu}(p, q)\big) \implies \\
		\big(conset^S_{D,\varepsilon,\mu}(V) \implies conset^T_{D,\varepsilon,\mu}(V)\big)
	\end{array}
\end{equation}

Monotoniczność gęstościowej łączności nie jest wykorzystywana bezpośrednio, ale na jej podstawie, można dojść do wniosku, że jeśli zbiór $ V $ nie jest gęstościowo połączony$^T$, to nie jest też gęstościowo połączony$^S $. Posiłkując się faktem, że dla dowolnej podprzestrzeni $ U $ grupa$^U$ jest gęstościowo połączonym zbiorem$^U$, dochodzimy do wniosku, że jeśli nie istnieje żadna grupa$^T$, to nie może też istnieć żadna grupa$^S$. Tym samym, jeśli nie istnieje żadna grupa$^T$, to nie ma sensu wykonywać grupowania w podprzestrzeni $ S $.

Z \myhyperref{eq:d-vs-d-projected}{wyrażenia} wynika jeszcze jedna własność, którą SUBCLU wykorzystuje do ograniczenia liczby punktów zbioru danych podczas grupowania w podprzestrzeniach. Zbiór $ V $ może być zbiorem gęstościowo połączonym$^S$, tylko jeśli $ V $ jest podzbiorem zbioru gęstościowo połączonego$^T$ $ W $.
\begin{equation}
 conset^S_{D,\varepsilon,\mu}(V) \implies \exists_W\,V\subseteq W \land conset^T_{D,\varepsilon,\mu}(W)
\end{equation}
Dzięki temu grupowanie w podprzestrzeni $ S $ o wymiarowości większej niż jeden może być wykonywane na zbiorze danych składającym się tylko z punktów należących do grup znalezionych w podprzestrzeni $ T \subset S $.

Kompletny algorytm grupowania SUBCLU przedstawia \myhyperref{alg:subclu}{algorytm}.

\begin{algorithm}
 	\caption{SUBCLU \cite{subclu}}\label{alg:subclu}

	\DontPrintSemicolon
	
	\SetKwFunction{subclu}{subclu}
	\SetKwFunction{generateCandidates}{generageCandidateSubspaces}
	\SetKwFunction{dbscan}{dbscan}
	
	\setcounter{AlgoLine}{0}
	\nonl\SetKwProg{myproc}{Wejście}{}{}
	\myproc{}{
		$D$ - zbiór danych \;
		$\varepsilon $ - promień otoczenia \;
		$\mu $ - próg liczności otoczenia \;
	}
	\setcounter{AlgoLine}{0}
	\nonl\SetKwProg{myproc}{Wyjście}{}{}
	\myproc{}{
		grupy istniejące we wszystkich podprzestrzeniach zbioru $ D $\;
	}
	
	\setcounter{AlgoLine}{0}
	\nonl\SetKwProg{myproc}{Definicje}{}{}
	\myproc{}{
		$ C^S $ - zbiór grup w podprzestrzeni $ S $\;
		$ S_k $ - zbiór wszystkich $ k $ wymiarowych podprzestrzeni, w których istnieje przynajmniej jedna grupa\;
		$ C_k $ - zbiór wszystkich zbiorów grup w podprzestrzeniach o wymiarowości $ k $, $ \set{C^S | |S|=k} $ \;
	}
	\nonl\SetKwProg{myalg}{Algorytm}{}{}
	\myalg{\subclu{$D$, $\varepsilon$, $\mu$}}{
		$ S_1 \gets \emptyset $, 
		$ C_1 \gets \emptyset $\;
		\ForEach{$ a \in A_D $}{
			$ C^{\{a\}} \gets $ \dbscan{D, $\{a\}$, $\varepsilon$, $\mu$}\;
			\If{$ C^{\{a\}} \neq \emptyset $}{
				$ S_1 \gets S_1 \cup \{a\}$\;
				$ C_1 \gets C_1 \cup C^{\{a\}} $\;		
			}
		}
		$ k \gets 1 $\;
		\While{$ C_k \neq \emptyset$}{
			$ CandS_{k+1} \gets $ \generateCandidates{$ S_k $}\;
			\ForEach{$ cand \in CandS_{k+1} $}{
				$ bestSubspace \gets \min\limits_{s\in S_k \land s\subseteq cand} \sum_{C_i \in C^2}|C_i|$\;
				$ C^{cand} \gets \emptyset $\;
				\For{$ cl \in C^{bestSubspace} $}{
					$ C^{cand} \gets C^{cand}\ \cup $ \dbscan{cl, cand, $\varepsilon$, $\mu$}\;
					\If{$ C^{cand} \neq \emptyset $}{
						$ S_{k+1} \gets S_{k+1}\cup cand $\;
						$ C_{k+1} \gets C_{k+1} \cup C^{cand} $\;
					}
				}
			}
			$ k \gets k + 1 $
		}
	}
	\setcounter{AlgoLine}{0}
	\nonl\SetKwProg{myproc}{Procedura}{}{}
	\myproc{\generateCandidates{$ S_k $}}{
		$ CandS_{k+1} \gets \set{S |T,U\in S_k \land S = T \cup U \land |S|=k+1} $ \;
		$ CandS_{k+1} \gets \set{cand \in CandS_{k+1} | \neg \exists_{S \subset cand}\,|S|=k \land S\notin S_k} $ \;
		\KwRet{$ CandS_{k+1} $}
	}
	\setcounter{AlgoLine}{0}
	\nonl\SetKwProg{myproc}{Procedura}{}{}
	\myproc{\dbscan{$ D $, $S$, $\varepsilon$, $\mu $}}{
		\tcc{Zwraca grupy według definicji 8. Można wykorzystać odpowiednio zmodyfikowany \myhyperref{alg:dbscan}{algorytm}. }
	}
\end{algorithm}

\subsection{Wydajność}
Wydajność SUBCLU jest ściśle uzależniona od wydajności implementacji algorytmu DBSCAN zastosowanego do grupowania w poszczególnych podprzestrzeniach. Równie ważny jest dobór parametrów $ \varepsilon $ i $ \mu $, który wpływa na liczbę i wielkość grup i stąd na liczbę podprzestrzeni, w których przeprowadzane jest grupowanie algorytmem DBSCAN. W pesymistycznym przypadku grupowanie algorytmem DBSCAN może zostać przeprowadzone w $ 2^{dim(D)}-1 $ podprzestrzeniach. Jeśli założymy, że złożoność DBSCAN to $\mathcal{O}( dim(D)|D|^2 )$, to pesymistyczna złożoność SUBCLU wyniesie $ \mathcal{O}(2^{dim(D)}dim(D)|D|^2) $.