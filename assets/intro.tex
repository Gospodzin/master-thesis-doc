\chapter{Wstęp}
Wraz ze wzrostem mocy obliczeniowej komputerów rośnie liczba zastosowań, jakie znajdują. Jedną z dynamicznie rozwijających się obecnie dziedzin jest tak zwana eksploracja danych. Możliwości dzisiejszych komputerów są ogromne, a infrastruktura pozwalająca na przetwarzanie terabajtów, a nawet petabajtów danych jest w zasięgu niemal każdego przedsiębiorstwa i instytucji. Przetwarzanie takich ilości danych stwarza dla biznesu i nauki nowe możliwości. Na podstawie zebranych danych o klientach czy produktach można uzyskać cenną wiedzę, która pozwala być konkurencyjnym na rynku. Mnogość zastosowań przetwarzania danych można znaleźć w medycynie, astronomii i innych dziedzinach nauki. \par
Dane analizuje się za pomocą metod, pośród których wyróżnia się między innymi poszukiwanie asocjacji, regresję, klasyfikację i grupowanie. W swojej pracy zajmuję się tematem grupowania gęstościowego, które jest szczególnym rodzajem grupowania. Badania, które przeprowadziłem, dotyczą algorytmów DBSCAN oraz SUBCLU. Celem mojej pracy było przeanalizowanie istniejących metod optymalizacji wydajnościowej oraz przedstawienie własnych propozycji usprawnień i rozwiązań. \par
W ramach pracy dyplomowej powstało również narzędzie, które pozwala grupować dane wybranymi algorytmami i następnie zwizualizować wyniki. W ostatnim rozdziale czytelnik znajdzie opis interfejsu oraz możliwości powstałego programu.