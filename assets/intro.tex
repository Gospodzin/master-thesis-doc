\subsection*{Wstęp}
W tym semestrze zajmowałem się głownie trzema zagadnieniami: 
\begin{itemize}
	\item[a)]{optymalizacją grupowania danych w jedno-wymiarowej przestrzeni,}
	\item[b)]{znalezieniem optymalnego położenia punktu referencyjnego dla dodatnich znormalizowanych danych,}
	\item[c)]{zrównolegleniem DBSCAN.}
\end{itemize}
Dalej w dokumencie zostały opisane zagadnienia a, b. Punkt c został opisany częściowo, powstał jednak algorytm, który pozwala na zrównoleglenie grupowania, a jego implementacja znajduje się w dostarczonym programie. Okazuje się, że w niektórych przypadkach działa wyjątkowo wydajnie.