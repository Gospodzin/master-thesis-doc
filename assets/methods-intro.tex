W tym rozdziale zostaną omówione techniki pozwalające na wydajne grupowanie opisanymi w poprzednim rozdziale algorytmami. Największe pole do optymalizacji daje wyznaczanie $ \varepsilon $-otoczenia. Istnieje wiele metod indeksacji danych usprawniających wyszukiwanie $ \varepsilon $ otoczenia. Przedstawię metodę nierówności trójkąta \cite{tidbscan,cosbyeuc} oraz metodę projekcji \cite{tivsp}. Opisałem również trzy autorskie optymalizacje: wydajne jednowymiarowe grupowanie, wybór efektywnego punktu referencyjnego dla nieujemnych, znormalizowanych wektorów i zrównoleglenie DBSCAN.